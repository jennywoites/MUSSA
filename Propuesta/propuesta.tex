\documentclass[a4paper]{article}

%%%%% Paquetes

\usepackage[spanish]{babel} 
\usepackage[utf8]{inputenc}
\usepackage{makeidx} %Indice
\usepackage{amsmath} %Ecuaciones sin numero al costado
\usepackage{graphicx} %Imagenes
\usepackage{fancyhdr} %Para encabezado y pie de pagina

\usepackage{hyperref} %Para url y links
\hypersetup{colorlinks=false}

\usepackage{longtable} %Para tablas que ocupan mas de una hoja

%%%%% Paquetes


\begin{document}

\pagenumbering{gobble} %No numerar la primer pagina

\begin{titlepage}
	\centering
	\includegraphics{Imagenes/logo_fiuba.png}\par\vspace{1cm}
	{\scshape\LARGE Facultad de Ingeniería \par
	Universidad de Buenos Aires  \par}
	\vspace{1.5cm}
	{\Large\bfseries Propuesta de Trabajo Profesional\par}
	\vspace{1.5cm}
	{\huge\bfseries MUSSA \par}
	\vspace{0.5cm}
	{\huge\bfseries Generador de Planes de Carrera personalizados utilizando Programación Lineal Entera \par}
	\vspace{1cm}
	{\Large\itshape Jennifer Andrea Woites\par}
	\vfill
	{\Large
	Tutor: Lic. Rosa Wachenchauzer \par
	\vspace{0.3cm}
	Co-Tutor: Ing. Diego Essaya}
	\vfill
% Bottom of the page
	{\large Septiembre, 2017 \par}
\end{titlepage}

  \newpage
  
  \pagenumbering{arabic} %Comenzar a numerar las paginas

  \tableofcontents % Indice
    
%%%% Cabecera y pie de pagina
  \pagestyle{fancy} % seleccionamos un estilo
  \lhead{Jennifer Andrea Woites - Padrón: 93274}
  \rhead{}
  \renewcommand{\headrulewidth}{0.4pt} % grosor de la línea de la cabecera
%%%% Cabecera y pie de pagina

  \newpage


\section{Introducción}

El siguiente documento presenta la propuesta del Trabajo Profesional de Ingeniería en Informática de la estudiante Jennifer Andrea Woites, padrón 93274. Los docentes que estarán a cargo son: Lic. Rosa Graciela Wachenchauzer como tutora del trabajo profesional, y el Ing. Diego Essaya como cotutor. \newline

El objetivo del proyecto es aplicar los conocimientos adquiridos en la carrera. \newline
MUSSA - Generador de Planes de Carrera personalizados utilizando Programación Lineal Entera". \newline

El objetivo general del presente trabajo, será la realización de una web responsive que permita a los alumnos la visualización de sus materias, la administración de las mismas, la utilización de encuestas y la generación de un plan de carrera en base a parámetros personalizados. Cabe destacar, que el corazón de este trabajo reside en la modelización y generación del algoritmo que permita la generación del plan de carrera personalizado; para ello, se generará un modelo de resolución con Programación Lineal Entera.

\subsection{Motivación}

Durante los años en que el alumno lleva adelante la carrera, muchas veces se ve obligado a no seguir con las materias tal como figuran en el plan por diversos motivos: cupos llenos en las cursos que prefiere, decisión para cursar con otros compañeros, necesidad de cursar menos materias por cuatrimestre, incompatibilidad laboral, razones personales, etc. Estos alumnos se ven forzados a tener que rediseñar el plan para poder acomodar las materias en un tiempo razonable de realización, que además sea compatible con las preferencias que él o ella tengan.

A esto, se suman materias que solo se dictan en el primer o segundo cuatrimestre, las correlatividades, entre otras, que dificultan la decisión de qué materia realizar antes que otra.

Luego, se debe tener en cuenta que en muchos de los planes de estudio, las materias electivas no están clasificadas por ramas o intereses y que es parte de las decisiones que debe tomar el alumno el elegir qué materias desea cursar, muchas veces decidiendo más por el nombre que por el contenido en sí porque no necesariamente sabe si le está aportando conocimiento en el área a la que le gustaría dedicarse. En otros casos, por comentarios de algunos compañeros que previamente cursar dichas materias, es capaz de asesorarse y decidir con un razonamiento más amplio.

Actualmente, la decisión de qué materia se cursa primero y cuál después, se toma "manualmente", es decir, se observa el punto en el que se está parado y se decide en base a lo mejor que se puede hacer en el corto plazo ya que es muy difícil observar el panorama completo cuanto más lejos de la meta se está.

En el marco de re inserción de alumnos que han abandonado la carrera y que le restan pocas materias por recibirse, es útil poder contar con una herramienta que priorice las materias en base a los tiempos que esta persona tiene disponible y tratar de recomendarle los mejores cursos para que pueda recibirse con prontitud, de forma de poder incrementar el número de profesionales recibidos en Argentina.

Se busca entonces, poder facilitar estas tareas a través de una plataforma web que permita obtener la información requerida para clasificar las materias (por ejemplo, a través de las encuestas), que facilite la generación de las notas para trámites como pedidos de créditos o excepciones de correlatividades que han de presentarse en la facultad posteriormente, y que, a través de parámetros configurados por el usuario, pueda armar automáticamente un plan de carrera que se ajuste a sus necesidades, pudiendo ser mutado si las mismas cambian con el paso del tiempo.

\section{Alcance}

El alcance de este trabjo contempla:

\begin{enumerate}
	\item Desarrollar una página web responsive que conste de los siguientes:
		\begin{itemize}
			\item Login / Sign In
			\item Almacenamiento y edición de datos personales, carreras, materias aprobadas
			%\item Excepciones de correlatividades
			%\item Simultaneidad de carreras
			\item Encuestas de cursos / materias
			\item Busquedas de materias con filtros
		\end{itemize}
		
	\item Modelar y desarrollar el algoritmo para la generación del plan de estudios personalizada
\end{enumerate}

\section{Características del Trabajo}

\subsection{Login / SigIn}

Para el Login / Sign In se contará con un usuario y contraseña que servirá como identificación. El padrón será un dato no obligatorio, de forma tal que pueda ser utilizado por aquellos alumnos que aún se encuentran en el CBC y no cuentan con padrón. Sin embargo, en caso de introducir el padrón, este deberá ser único ya que no puede haber dos alumnos con el mismo padrón.

Se debe permitir además, el recupero de contraseña a través de el mail proporcionado por el usuario, generando una clave random y enviándosela por mail. Esta clave deberá estar marcada como caducada  el usuario tendrá que modificarla en cuanto vuelva a entrar al sistema.

Para la primer implementación, no se realizará bloqueo y desbloqueos de usuarios.

\subsection{Perfil de usuario}

Cada usuario deberá tener sus datos personales (nombre, apellido, mail, padrón, etc), pero además, sus datos académicos que deberán incluir:

\begin{itemize}
	\item Carreras en las que se encuentra inscripto
	\item Materias aprobadas
	\item Materias en curso
	\item Materias con final pendiente
\end{itemize}

Se debe permitir además, que el alumno pueda descargar en pdf el listado de materias que tiene aprobadas (este listado suele ser solicitado en la mayoría, sino todos,los trámites de la facultad).

\subsection{Carreras de la facultad}

Las diferentes carreras de la facultad están compuestas por una serie de materias obligatorias, un trabajo profesional, posibilidad de un idioma obligatorio, posibilidad de rendir un examen de suficiencia de idioma, materias electivas, trabajo profesional, tesis, etc.

Para esta primer implementación se cargarán y modelarán los planes de estudios de Ingeniería en Informática y Licenciatura en Análisis de Sistemas (1986, plan 'viejo'). Posteriormente en futuras implementaciones, se añadirán las carreras restantes.

Cada materia tiene como datos su código, el nombre, los contenidos mínimos, los diferentes cursos disponibles con sus docentes y horarios, la cantidad de créditos y sus correlatividades.

Es posible modelar las diversas materias en un grafo dirigido no pesado, en el cual las aristas equivalen a las correlatividades. Es decir, si la materia A es correlativa de B y C, entonces habrá una arista de A hacia B, y de A hacia C respectivamente.\newline


Los horarios de los cursos de las materias deberán mantener un histórico, y avisar en caso de que sea un curso nuevo (en cuyo caso se deberán repetir los horarios indicando en el cuatrimetsre al que se hayan copiado los horarios que podrían variar), se debe identificar las materias que se dictan solo el 1º o 2º cuatrimestre.

La obtención de los horarios del cuatrimestre actual se realizará a través del pdf publicado por la facultad cada cuatrimestre, basandose en un parser desarrollado por el LUGFI\cite{PARSER_LUGFI}.

\subsection{Excepciones de correlatividades}

Se deberá contar con una sección en la que sea posible ingresar pedidos de excepciones de correlatividad. Al hacer un pedido, se debe indicar la materia que se desea cursar y con qué materia se solicita la excepción, además del estado de la misma (si aún no fue cursada o si tiene el final pendiente). Adicionalmente podrá ingresar un texto con mayor detalle.

Al generar la solicitud de excepción, se guardará la misma en el historial de pedidos de excepciones en un estado 'No resuelta'. Además, se descargará en un pdf la nota modelo correspondiente ya completa con los datos necesarios para poder ser presentada en la facultad.

Desde el historial de excepciones será posible cambiar el estado a 'Aceptada' o 'Rechazada' y subir como adjunto la resolución correspondiente (esto permitirá tener en un único lugar todas las copias de las resoluciones que deben ser presentadas cuando se realiza el trámite del título).\newline


En caso de que el alumno no haya modificado el estado de la solicitud de excepción, cuando marque la materia como aprobada y deba completar la encuesta, se le añadirá una pregunta que requiera indicar si hizo la materia con la excepción o no.\newline


El manejo de las excepciones es importante para este sistema ya que las materias del plan de estudio de una determinada carrera tienen una correlatividad que podemos llamar "fuerte", es decir, no es posible cursarlas si no se tiene aprobada su correlativa; si se los coloca en un grafo, sería su predecesor. Lo que se desea modelar en ese caso es que si una excepción de correlatividad fue entregada con frecuencia con la correlativa no cursada aún (por ejemplo, las aprobadas son de al menos el doble que las rechazadas o similar), esa materia tendría una correlatividad débil con la materia con la cual se pidió excepción, por ello, se podría quitar el predecesor (con una alerta que avise que es requerido el pedido de excepción antes de cursar la materia).\newline


En la visualización de las materias, se incluirá el porcentaje de gente que la hizo con excepción de correlatividad con fines estadísticos.

\subsection{Simultaneidades de carreras}

Para los casos de simultaneidades de carrera se deberá formular un único plan combinado o permitir que seleccione el armado de un plan por carrera (por ejemplo si decidió terminar todas las materias de una carrera y luego terminar la otra), para este último se deberá indicar si comienza una a continuación de la otra o si comienza en el siguiente período lectivo anual.

Además, se deberán tener en cuenta: 

\begin{enumerate}
	\item Las equivalencias automáticas de materias: En caso de que se curse una materia con equivalencia, la equivalente debe darse por aprobada y no tenerse en cuenta para la distribución horaria en el plan de carrera.
	
	\item Equivalencias por resolución: Se debe permitir ingresar un pedido de equivalencia entre materias y/o por créditos. Luego, se deberá indicar si fue otorgada o no, anexando la correspondiente resolución e indicando los créditos o la materia dada por equivalencia, además de texto libre en caso de requerirlo.
	Si se cursa una materia que fue disparadora de una resolución por equivalencia, se deberá generar la alerta correspondiente en el cuatrimestre que corresponda, para poder generar automáticamente una nota de solicitud de equivalencia con las resoluciones históricas correspondientes. No se deberá tener en cuenta horarios de cursada de dicha materia pero contará para los créditos requeridos en el armado del plan.
	Se debe tener en cuenta que no es posible tener una equivalencia de una materia que fue otorgada por equivalencia (es decir si con la materia A dan B, y con la materia B dan C no es válido que solo con A se tengan B y C; solo se tendrá B).
	
	\item Algunas materias de mismo código tienen correlatividades diferentes según la carrera, por lo que deberán agregarse como materias diferentes en el armado del plan, y añadir la restricción de que, en caso de que se curse esta materia, solo se podrá hacer una de ellas, es decir o se cursa con las correlatividades de la carrera A o con las de la B, pero solo se hará una.
\end{enumerate}

\subsection{Encuestas}

Se deberá contar con una sección de encuestas a las que pueda puntuar a cada materia con el horario real de cursada que tuvo la materia, la cantidad de horas extras además de la cursada que requiere la materia por semana, un puntaje del curso, la dificultad de los temas, la dificultad de los TPs y exámenes, preguntas acerca de la materia, de los profesores y comentarios personales.

Además, se añadirá una sección de Tema Principal para las materias de carácter de "electiva", con el que se podrá clasificar la materia para determinar si está relacionada con el Data Mining, Robótica, Gestión, etc de forma que pueda ser utilizado como preferencia para priorizar una materia electiva por sobre otra en el armado del plan. Respecto de esto, se busca que la cantidad de materias electivas que se seleccionen automáticamente en el plan correspondan a porcentajes de 'afinidad' que establezca el alumno con cada uno de estos temas.

Adicionalmente, se podrán indicar hasta 3 palabras claves (tags) con las que el alumno podría identificar la materia. Las 3 palabras más elegidas, serán las utilizadas para la agilización de búsquedas de materias puntuales.

Cada alumno tendrá su listado de encuestas realizadas y podrá contestar una sola vez la combinación [curso + cuatrimestre + año].

Los resultados de las encuestas serán públicos sin mostrar quién fue el usuario que respondió la encuesta, de forma tal que los docentes puedan hacer uso de la página para poder evaluar sus propios cursos.

Los valores numéricos de las encuestas serán tomados en promedio para evaluar la calidad del curso y la dificultad, parámetros que podrán ser utilizados luego como parte de las preferencias de armado del plan.

\subsection{Busqueda de Materias}

Desde la vista pública, se podrá acceder al buscador de materias. En él se podrá buscar por lo siguiente:

\begin{itemize}
	\item Carrera
	\item Código de materia
	\item Nombre (o parte del nombre) de la materia
	\item Palabras clave (tags más frecuentes de cada materia)
\end{itemize}

Para cada materia se podrán acceder a los datos generales de la misma y al historial de encuestas por cada cuatrimestre.

\subsection{Algoritmos para la generación del plan de estudios personalizado}

	Se analizó la posibilidad de realizar fuerza bruta con poda en primera instancia, pero la cantidad de árboles generados para analizar crecía con mucha rapidez.
Para poder generar los árboles, se partían de las materias disponibles (todas aquellas para las cuales las correlativas habían sido aprobadas) y de la cantidad de materias máxima que se permitía cursar por cuatrimestre. Para cada materia, se consideraba que tenía un único horario y que, sin importar con qué materias compartiera el cuatrimestre, sus horarios iban a ser compatibles.

Luego, se toma una de las materias al azar y se generaban dos arboles posibles, el primero con esta materia incluida en el cuatrimestre que se estaba intentando completar, el segundo sin esta materia y marcándola como no disponible para ser utilizada en este cuatrimestre. Luego, se proseguía con ambos árboles, manteniendo las materias disponibles por separado.

Los árboles que desencadenaban en una imposibilidad de continuar por no quedar materias disponibles (ej, todas fueron marcadas como no válidas para ese cuatrimestre) eran descartados.

De esta forma, para pocas materias, era posible conseguir en poco tiempo los planes de estudio posibles, y luego, elegir de estos el de menor duración en cuatrimestres. Ahora bien, para una cantidad de materias de la magnitud de un plan de estudios general, no es posible realizar por fuerza bruta con poda los posibles planes de estudio en un tiempo pequeño, ya que la cantidad de ramificaciones se incrementaba en $2^{n}$.\newline


Otra posibilidad que se analizó es la generación a través de un algoritmo Greedy, que sencillamente elija las materias en base a lo que, según números como la cantidad de créditos que añade, la cantidad de correlatividades que libera, etc, seleccione las materias más convenientes para el cuatrimestre que se esté analizando. El problema con esto, es que no realiza un análisis del problema global sino localizado, y si bien puede tener una buena solución, no necesariamente será la óptima, ya que no dista mucho de lo que hoy realizan los alumnos intuitivamente para decidir su cuatrimestre.\newline


En base a la investigación, se encontró que en algunas universidades han desarrollado sistemas para determinar las aulas y los horarios de los cursos que se dictarán utilizando <<Programación Lineal Entera>>, que es la línea que se seguirá en este trabajo.

Básicamente, se plantean restricciones que llamaremos "fuertes"  que consisten en restricciones que si o sí deben cumplirse (por ejemplo, no es posible estar en dos lugares al mismo tiempo, o que todas las materias obligatorias deben cumplirse), y por otro lado, restricciones que son "deseables", es decir, que cumplirlas hace que la solución elegida sea mejor, pero si no pueden cumplirse, es posible relajarlas pero puntuando negativo a la solución. De esta forma, se permite flexibilidad al problema. Una posible restricción deseable es que el puntaje del curso sea mayor a un determinado valor.

Así, se generan algunas de las siguientes ecuaciones básicas\footnote{Se han seleccionado solo las ecuaciones básicas del problema para ilustrar el trabajo a realizar. Falta el desarrollo de las ecuaciones de horarios, restricciones deseables, etc.}:\newline

Sean:\newline
$Y_{ij}$: La materia i se realiza en el cuatrimestre j [Boolean]\newline
$C_{i}$: Numero de cuatrimestre en que se realiza la materia i [Integer]\newline
M: Constante. Máxima cantidad de cuatrimestres posibles. \newline

Tenemos algunas de las siguientes restricciones fuertes que comienzan a modelar el problema:

\begin{itemize}
	\item La materia i obligatoria, debe cursarse en algun cuatrimestre. Además, este cuatrimestre debe ser único.
\end{itemize}

$\sum_{i=1}^{M} Y_{ij} = 1;  \forall i \in [Materias]$

\begin{itemize}
	\item Se debe conocer el numero de cuatrimestre en que es cursada la materia i.
\end{itemize}

$C_{i} = \sum_{j=1}^{M} j * Y_{ij};  \forall i \in [Materias]$

\begin{itemize}
	\item Si la materia B tiene como correlativa a A, entonces, la materia A debe realizarse primero. Lo que quiere decir, que el cuatrimestre de A debe ser menor que el número de cuatrimestre de la materia B. 
\end{itemize}

$C_{A} < C_{B}$; o en general:

$C_{i} < C_{j};  \forall$ j que tenga como correlativa a la materia i

\begin{itemize}
	\item La cantidad de materias por cuatrimestre no puede superar un valor máximo preestablecido por el usuario. (cantidad de materias máxima que el alumno está dispuesto a cursar por cuatrimestre).
\end{itemize}

$\sum_{\forall i \in MATERIAS} Y_{ij} \leq$ Max\_cant\_materias\_cuatrimestre;  $\forall$ j $\in$ [1,M]

\begin{itemize}
	\item En el caso básico la función objetivo que se busca minimizar es la cantidad de cuatrimestres totales. Para ello, se busca el valor más grande de los números de cuatrimestres. Posteriormente, esta función objetivo se modificará para ser penalizada por otras restricciones.
\end{itemize}

$Z_{min}$ = Max($C_{i}$);  $\forall$ i $\in$ [Materias]

\subsection{Armado y Visualización del Plan de Carrera}

Para permitir el armado del plan, se deberán poder establecer las preferencias del alumno. Estas preferencias serán guardas y precargadas cada vez que el alumno desee volver a diseñar el plan por algún motivo (por ejemplo, no completó las materias que se proponía cursar y requiere ajustar las materias que realmente puede cursar el cuatrimestre siguiente, o que comenzó a trabajar y por ende, tiene menos disponibilidad horaria, etc).

Para ello clasificaremos las preferencias en fuertes [F] (la restricción debe cumplirse) o deseables [D] (si no se cumplen, se pondera negativamente):

\begin{itemize}
	\item Cantidad de cuatrimestres máximos de duración del plan [F]
	\item Máxima cantidad de horas de cursada por semana [F]
	\item Máxima cantidad de horas de trabajo extra además de la cursada, por semana [F]
	\item Días y horarios que el alumno tiene disponibles para cursar [F]
	\item Preferencias de topics de las materias electivas [D]: Seleccionar los porcentajes deseados para cada Topic.
	\item Puntuación del curso mayor a un determinado valor [D]	
\end{itemize}

Una vez confeccionado el plan, éste debe poder visualizarse tanto en una grilla como en forma gráfica en la que se muestren las correlatividades entre las materias que se van a cursar.

\section{Tecnologías y Herramientas}

\subsection{Librería: PuLP}

PuLP es una librería Open Source escrita en Python.

Es usada para describir problemas de optimización como modelos matemáticos. PuLP puede llamar a numerosos solvers de programación lineal, tales como CBC, GLPK, CPLEX, Gurobi, etc, para resolver el modelo y luego utilizar comandos de Python para manipular y mostrar el resultado de la solución.

\subsection{Herramientas}

\begin{table}[htbp]
\begin{center}
\begin{tabular}{|l|l|}
\hline
\textbf{Categoría}				 			& 		\textbf{Herramientas} \\
\hline
Backend										&		Python, Flask \\
\hline
Frontend									&		JavaScript, HTML, CSS/SASS, Bootstrap \\
\hline
Base de Datos			 					&		SQLAlchemy\footnote{Python SQL toolkit and Object Relational Mapper} \\
\hline
Entornos de Desarrollo						& 		Sublime \\
\hline
Control de Versiones						& 		GitHub \\
\hline
Administración y Control del Proyecto		& 		GDocs, Trello \\
\hline
Elaboración de Documentos					& 		LaTex \\
\hline
Librerías Principales						& 		PuLP \\
\hline
\end{tabular}
\end{center}
\end{table}

\section{Plan de Trabajo}

\subsection{Equipo de Trabajo}

El equipo de trabajo está compuesto por:

\begin{itemize}
	\item Tutor: Lic. Rosa Graciela Wachenchauzer
	\item Co-tutor: Ing. Diego Essaya
	\item Desarrollador: Jennifer Andrea Woites
\end{itemize}

\subsection{Metodología}

Para el desarrollo de este trabajo se seguirá con una metodología mixta con Kanban y Scrum.

Por un lado, se tendrá un tablero con las tareas principales como 'Pendientes', 'En curso', 'Desarrollo Terminado', 'En Testing', 'Finalizadas', y se las hará mover a través del tablero según corresponda; para poder tener una visión global del proyecto.

Cada una de estas tareas pincipales será explotada en varias subtareas o tareas secundarias. Cuando todas las tareas secundarias de una tarea hayan finalizado, se podrá mover la tarea principal a 'Desarrollo Terminado' de forma que después pueda ser tomada para un testing integrado y completo de las misma.

Cada una de las tareas secundarias se corresponderá con un issue en gitHub para facilitar el trackeo de los commits con respecto a la funcionalidad que se estaba desarrollando.

Por otro lado, se establecerá un período de tiempo (como un sprint) en el que se establecerán las tareas secundarias deseadas a desarrollar durante ese lapso. Al finalizar el sprint se realizará la entrega del avance que haya sido completado. Los sprints durarán un tiempo no mayor a dos semanas. En caso que durante el sprint se detecte que se debe modificar alguna tarea, se realizará el ajuste correspondiente y se intercambiará por otras menos prioritarias de la misma duración que las nuevas que se incorporan.

A medida que se vayan avanzando en los sprints se realizarán los ajustes que se consideren necesarios en base a lo que se establezca con el tutor y el cotutor en las correspondientes reuiniones.

\subsection{Estimaciones}

Se realizó una primera estimación con todas las tareas en conjunto. Todas las estimaciones se encuentran en horas reloj.

\begin{longtable}{| c | l | c |}

%Encabezado en primera hoja
\hline
\textbf{Código}		& 	\textbf{Descripción}											& 	\textbf{Horas estimadas} \\
\hline \hline
\endfirsthead

%Encabezado en el resto de las hojas
\hline
\textbf{Código}		& 	\textbf{Descripción}											& 	\textbf{Horas estimadas} \\
\hline \hline
\endhead

%Fondo de las hojas, excepto la ultima
%\multicolumn{2}{c}{Sigue en la página siguiente.}
\endfoot

%Fondo de la ultima hoja
\endlastfoot

%Cuerpo de la tabla

1					&	Propuesta + Investigación 										&	45\\
\hline
2					&	Configuración del Entorno 										&	10\\
\hline
3					&	LogIn / Sign In			 										&	30\\
\hline
3.1					&	LogIn					 										&	10\\
\hline		
3.2					&	Sign In					 										&	8\\
\hline
3.3					&	Recupero de Contraseña											&	6\\
\hline
3.4					&	Cambio de contraseña											&	6\\
\hline	
4					&	Perfil del Usuario												&	37\\
\hline
4.1					&	Datos personales												&	8\\
\hline
4.2					&	Carreras Inscripto												&	3\\
\hline
4.3					&	Materias Aprobadas / Desaprobadas								&	16\\
\hline
4.4					&	Materias en Curso												&	5\\
\hline
4.5					&	Materias con Final Pendiente									&	5\\
\hline
5					&	Carreras de la Facultad											&	18\\
\hline
5.1					&	Modelización de las Carreras									&	10\\
\hline
5.2					&	Carga de datos inicial de los planes de estudio				&	8\\
\hline
6					&	Excepciones de Correlatividad									&	38\\
\hline
6.1					&	Ingreso de pedidos de excepción								&	8\\
\hline
6.2					&	Historial de Excepciones										&	8\\
\hline
6.3					&	Impresión de solicitud de excepción de correlatividad			&	6\\
\hline
6.4					&	Modificación de correlatividad fuerte / débil					&	10\\
\hline
6.5					&	Carga de datos de prueba										&	6\\
\hline
7					&	Simultaneidades de Carrera										&	40\\
\hline
7.1					&	Generación de planes combinados								&	16\\
\hline
7.2					&	Equivalencias automáticas										&	8\\
\hline
7.3					&	Equivalencias de materia por resolución						&	8\\
\hline
7.4					&	Créditos otorgados por resolución								&	4\\
\hline
7.5					&	Carga de datos de prueba										&	4\\
\hline
8					&	Encuestas de cursos / Materias									&	80\\
\hline
8.1					&	Generación de encuestas											&	16\\
\hline
8.2					&	Visualización de las encuestas (con y sin login)				&	24\\
\hline
8.3					&	Llenado de encuestas											&	16\\
\hline
8.4					&	Historial de encuestas completadas								&	8\\
\hline
8.5					&	Búsqueda de materias											&	16\\
\hline
9					&	Horarios de los cursos											&	32\\
\hline
9.1					&	Sincronización													&	16\\
\hline
9.2					&	Historial														&	8\\
\hline
9.3					&	Detección de nuevos cursos										&	3\\
\hline
9.4					&	Cursos dictados en un único cuatrimestre						&	5\\
\hline
10					&	Confección del plan de Carrera Personalizado					&	116\\
\hline
10.1				&	Selección de Preferencias										&	12\\
\hline
10.2				&	Notificacion													&	24\\
\hline
10.3				&	Algoritmo de confección del plan								&	80\\
\hline
11					&	Visualización del plan de Carrera								&	40\\
\hline
11.1				&	Modelo Gráfico													&	30\\
\hline
11.2				&	Modo Texto														&	10\\
\hline
12					&	Documentación													&	24\\
\hline
13					&	Reuniones														&	25\\
\hline
					&	\textbf{SUBTOTAL}												&	\textbf{535}\\
\hline
14					&	Testing															&	107\\
\hline
15					&	Administración y Control del Proyecto							&	27\\
\hline
					&	\textbf{TOTAL}													&	\textbf{669}\\
\hline
\\ % esta línea es importante para que deje un espacio entre la tabla y el nombre de la tabla.
\caption{Estimaciones con alcance completo}
\label{ta:estimaciones_alcance_completo}
\end{longtable}


Luego de la primer estimación, se decidió reducir el alcance para la primer implementación, la cual consistirá en el alcance del trabajo profesional, sin embargo es de interés poder luego de finalizado el mismo, coordinar con los alumnos del LUGFI para poder continuar con el desarrollo de las funcionalidades que han quedado fuera del alcance y de esta forma, realizar un desarrollo continuo en el tiempo junto a otros estudiantes de la Universidad.

\begin{longtable}{| c | l | c |}

%Encabezado en primera hoja
\hline
\textbf{Código}		& 	\textbf{Descripción}											& 	\textbf{Horas estimadas} \\
\hline \hline
\endfirsthead

%Encabezado en el resto de las hojas
\hline
\textbf{Código}		& 	\textbf{Descripción}											& 	\textbf{Horas estimadas} \\
\hline \hline
\endhead

%Fondo de las hojas, excepto la ultima
%\multicolumn{2}{c}{Sigue en la página siguiente.}
\endfoot

%Fondo de la ultima hoja
\endlastfoot

%Cuerpo de la tabla

1					&	Propuesta + Investigación 										&	45\\
\hline
2					&	Configuración del Entorno 										&	10\\
\hline
3					&	LogIn / Sign In			 										&	30\\
\hline
3.1					&	LogIn					 										&	10\\
\hline		
3.2					&	Sign In					 										&	8\\
\hline
4					&	Perfil del Usuario												&	37\\
\hline
4.1					&	Datos personales												&	8\\
\hline
4.3					&	Materias Aprobadas / Desaprobadas								&	16\\
\hline
5					&	Carreras de la Facultad											&	18\\
\hline
5.1					&	Modelización de las Carreras									&	10\\
\hline
5.2					&	Carga de datos inicial de los planes de estudio				&	8\\
\hline
8					&	Encuestas de cursos / Materias									&	80\\
\hline
8.1					&	Generación de encuestas											&	16\\
\hline
8.2					&	Visualización de las encuestas (con y sin login)				&	24\\
\hline
8.3					&	Llenado de encuestas											&	16\\
\hline
8.4					&	Historial de encuestas completadas								&	8\\
\hline
8.5					&	Búsqueda de materias											&	16\\
\hline
9					&	Horarios de los cursos											&	32\\
\hline
9.1					&	Sincronización													&	16\\
\hline
9.4					&	Cursos dictados en un único cuatrimestre						&	5\\
\hline
10					&	Confección del plan de Carrera Personalizado					&	116\\
\hline
10.1				&	Selección de Preferencias										&	12\\
\hline
10.3				&	Algoritmo de confección del plan								&	80\\
\hline
11					&	Visualización del plan de Carrera								&	40\\
\hline
11.2				&	Modo Texto														&	10\\
\hline
12					&	Documentación													&	24\\
\hline
13					&	Reuniones														&	25\\
\hline
					&	\textbf{SUBTOTAL}												&	\textbf{367}\\
\hline
14					&	Testing															&	74\\
\hline
15					&	Administración y Control del Proyecto							&	19\\
\hline
					&	\textbf{TOTAL}													&	\textbf{460}\\
\hline
\\ % esta línea es importante para que deje un espacio entre la tabla y el nombre de la tabla.
\caption{Estimaciones con alcance reducido}
\label{ta:estimaciones_alcance_reducido}
\end{longtable}


\subsection{Cronograma}

A continuación se hace un cronograma de entregables tentativo al finalizar cada una de las iteraciones. Las mismas pueden ser modificadas en caso de verse necesario, por el tutor, el cotutor o el desarrollador. La fecha de cada una será determinada en conjunto con el tutor y el cotutor.

---------- FALTA COMPLETAR -------------

\newpage
\section{Referencias y Material consultado}

\renewcommand\refname{\small}

\begin{thebibliography}{X}

% BEGIN Referencia
\bibitem{PLE_HORARIOS_UNIV} \textit{<<Programación de Horarios de Clases y Asignación de Salas para la Facultad de Ingeniería de la Universidad Diego Portales Mediante un Enfoque de Programacion Entera>>}

\textsc{Rodrigo Hernandez, Jaime Miranda P. y Pablo A. Rey}

\url{http://www.dii.uchile.cl/ris/RISXXII/horariosUDP_RISVersion%20FINAL.pdf}
% END Referencia


% BEGIN Referencia
\bibitem{MODELOS_PLE} \textit{<<Formulación y Resolución de Modelos de Programación Matemática en Ingeniería y Ciencia>>}

\textsc{Enrique Castillo, Antonio J. Conejo, Pablo Pedregal, Ricardo García y Natalia Alguacil}

\url{http://www.dia.fi.upm.es/~jafernan/teaching/operational-research/LibroCompleto}
% END Referencia


% BEGIN Referencia
\bibitem{PULP_DOC} \textit{<<PuLP Classes>>}

\url{https://www.coin-or.org/PuLP/pulp.html}
% END Referencia


% BEGIN Referencia
\bibitem{PULP_PYTHON} \textit{<<Linear Programming with Python and PuLP>>}

\textsc{Ben Alex Keen}

\url{http://benalexkeen.com/linear-programming-with-python-and-pulp/}
% END Referencia


% BEGIN Referencia
\bibitem{PULP_DOC_OFICIAL_PYTHON} \textit{<<Optimization with PuLP>>}

\textsc{Stuart Mitchell, Anita Kean, Andrew Mason, Michael O’Sullivan, Antony Phillips}

\url{https://pythonhosted.org/PuLP/}
% END Referencia


% BEGIN Referencia
\bibitem{PARSER_LUGFI} \textit{<<Parser de Materias para el Organizador de Horarios>>}

\textsc{LUGFI}

\url{https://github.com/lugfi/organizador-fiuba/blob/master/tools/parser.js}
% END Referencia


% BEGIN Referencia
\bibitem{SQL_ALCHEMY} \textit{<<SQL Alchemy>>}

\url{https://www.sqlalchemy.org/}
% END Referencia


% BEGIN Referencia
\bibitem{FLASK_TUTORIAL} \textit{<<The Flask Mega-Tutorial>>}

\textsc{Miguel Grinberg}

\url{https://blog.miguelgrinberg.com/post/the-flask-mega-tutorial-part-i-hello-world}
% END Referencia


\end{thebibliography}

\end{document}